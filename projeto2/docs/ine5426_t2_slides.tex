\documentclass{beamer}

\usepackage[brazil]{babel}
\usepackage[T1]{fontenc}
\usepackage[utf8]{inputenc}

\usetheme{CambridgeUS}
\usecolortheme{seagull}
\setbeamertemplate{headline}{}
\setbeamertemplate{itemize items}[triangle]
\setbeamertemplate{footline}[frame number]{}
\setbeamertemplate{navigation symbols}{}

\title{
    Novas funcionalidades para Łukasiewicz
}
\author{
    Douglas Martins\thanks{arquiteto de sistema}
    \and Gustavo Zambonin\thanks{gerente de projeto, projetista de linguagem}
    \and Marcello Klingelfus\thanks{testador}
}
\institute{
    Universidade Federal de Santa Catarina \\
    Departamento de Informática e Estatística \\
    INE5426 - Construção de Compiladores
}
\date{}

\begin{document}

\begin{frame}
    \titlepage
\end{frame}

\begin{frame}
    \frametitle{Primeira extensão: o paradigma funcional}

    \begin{itemize}
        \item Decomposição de um problema como um conjunto de funções
        \item Função anônima simples -- 
            {\footnotesize \texttt{lambda type arg:\;expr}}
        \item Função de alta ordem de mapeamento --
            {\footnotesize \texttt{map(function, list)}}
        \begin{itemize}
            \item aplica função em todos os elementos de um iterável
        \end{itemize}
        \item Função de alta ordem de redução --
            {\footnotesize \texttt{fold(function, list)}}
        \begin{itemize}
            \item reduz elementos de um iterável recursivamente para um
                tipo primitivo de acordo com uma função
        \end{itemize}
        \item Função de alta ordem de seleção --
            {\footnotesize \texttt{filter(function, list)}}
        \begin{itemize}
            \item remove elementos de um iterável que não respeitam uma
                função booleana
        \end{itemize}
    \end{itemize}

\end{frame}

\begin{frame}[fragile]
    \frametitle{Segunda extensão: os tipos \texttt{char} e \texttt{str}}

    \begin{itemize}
        \item \texttt{char} tem aspas simples, \texttt{str} tem aspas duplas
        \item operador de concatenação \texttt{+}
        \begin{itemize}
            \item \texttt{char + str}, \texttt{char + char} produz coerção 
        \end{itemize}
        \item \texttt{str} é similar a \texttt{char array},
            então suporta operação de índice
        \item adiciona também fácil suporte a comentários de uma linha com \#
    \end{itemize}

    \begin{verbatim}
        char letter = 'a'
        str word = "luka", sum
        # this is a comment discarded by the parser
        sum = letter + word(1)

        char var: letter = 'a'
        str var: word = "luka", sum
        = sum + [str] letter [index] word 1
    \end{verbatim}
\end{frame}

\begin{frame}[fragile]
    \frametitle{Integração com LLVM}

    \begin{itemize}
        \item Exemplo de geração de código intermediário:

        \begin{verbatim}
  int fun math() {        | define i64 @math() {
    int z = 0, y = 2      | entry:
    int x = z             |   %z = alloca i64
    x = y * 10 + x - 15   |   store i64 0, i64* %z
    z = x                 |   %x = load i64, i64* %z
    ret x                 |   %tmp2 = sub i64 %x, 15
  }                       |   %tmp = add i64 20, %tmp2
                          |   %x1 = add i64 %tmp, %x
                          |   store i64 %x1, i64* %z
                          |   ret i64 %x1
                          | }
        \end{verbatim}
    \end{itemize}
\end{frame}

\begin{frame}[fragile]
    \frametitle{\emph{Transpiling} para Python}

    \begin{itemize}
        \item Exemplo de tradução entre linguagens:

        \begin{verbatim}
  int fun math() {        | def math():
    int z = 0, y = 2      |     z = 0, y = 2
    int x = z             |     x = z
    x = y * 10 + x - 15   |     x = y * 10 + x - 15
    z = x                 |     z = x
    ret x                 |     return x
  }                       |
        \end{verbatim}
    \end{itemize}
\end{frame}

\end{document}
