\documentclass{article}

\usepackage[brazil]{babel}
\usepackage[T1]{fontenc}
\usepackage[a4paper, margin=1.5cm]{geometry}
\usepackage[colorlinks, urlcolor=blue, citecolor=red]{hyperref}
\usepackage[utf8]{inputenc}
\usepackage{enumitem}

\newenvironment{smallitem}{
    \vspace{-1mm}
    \begin{itemize}
    \setlength{\parskip}{0pt}
    \setlength{\itemsep}{2pt}
}{
    \vspace{-2mm}
    \end{itemize}
}

\newenvironment{smallenum}{
    \vspace{-1mm}
    \begin{enumerate}[label=\roman*.]
    \setlength{\parskip}{0pt}
    \setlength{\itemsep}{2pt}
}{
    \vspace{-2mm}
    \end{enumerate}
}

\title{\textbf{Proposta de extensão para a linguagem Łukasiewicz}}
\author{
    Douglas Martins \\
    {\small arquiteto de sistema}
    \and Gustavo Zambonin \\
    {\small gerente de projeto, projetista de sistema} \vspace{2mm}\\
    {\small Construção de Compiladores (UFSC -- INE5426)}
    \footnote{\texttt{
        \{marcelino.douglas,gustavo.zambonin,marcello.klingelfus\}@grad.ufsc.br
        \hfill \href{https://github.com/zambonin/ufsc-ine5426}{src/}
    }}
    \and Marcello Klingelfus \\
    {\small testador}
}
\date{}

\begin{document}

\maketitle

\section{Funções anônimas e de alta ordem}

\subsection{Definições}

Adiciona-se à linguagem suporte a funções anônimas (não depende de um
identificador para ser utilizada) e funções de alta ordem (toma uma ou mais
funções como parâmetro) denominadas \texttt{map}, \texttt{fold} e
\texttt{filter}. \medskip

Funções anônimas respeitam as seguintes restrições:
\begin{smallitem}
    \item \texttt{lambda} e a respectiva letra minúscula em grego,
        $\lambda$ (\texttt{U+03BB}), tornam-se palavras reservadas da linguagem;
    \item devem conter no mínimo um (1) argumento;
    \item a função pode ser utilizada da seguinte maneira:
        \texttt{$\lambda$($a_1$, $a_2$, $\dots$, $a_n$)}, onde $a_n$ é o
        $n$-ésimo argumento da função previamente declarada.
    \item uma função anônima pode ser descartada com a diretiva
        \texttt{$\lambda$()}, para que outra possa ser declarada;
    \item opera apenas sobre tipos básicos da linguagem (números inteiros
        e de ponto flutuante, booleanos e caracteres).
\end{smallitem}

Funções de alta ordem respeitam as seguintes restrições:
\begin{smallitem}
    \item \texttt{map}, \texttt{fold} e \texttt{filter} tornam-se palavras
        reservadas da linguagem;
    \item devem ter obrigatoriamente dois argumentos, nesta ordem: uma função
        anônima e um \texttt{array}, chamados respectivamente
        de \texttt{$\lambda$} e \texttt{it} neste documento para facilitar a
        visualização das regras;
    \item será necessário estender a linguagem com a adição de um operador
        unário \texttt{len}, que retorna o tamanho do \texttt{array}, para que
        o código gerado nas funções de alta ordem seja semanticamente correto;
    \item o tipo de retorno de \texttt{map} deve ser igual ao tipo de
        \texttt{it}, que deve ter tipo primitivo igual ao tipo de retorno
        de \texttt{$\lambda$};
    \item o tipo de retorno de \texttt{fold} deve ser igual ao de
        \texttt{$\lambda$} e ao tipo primitivo de \texttt{it};
    \item o tipo de retorno de \texttt{filter} deve ser igual ao tipo de
        \texttt{it}, e \texttt{$\lambda$} deve ser sempre uma função booleana;
    \item para evitar erros de redefinição, as funções de alta ordem
        levam como prefixo o nome de seu segundo parâmetro seguido de
        um traço inferior;
    \item arrays, índices e variáveis temporários devem ser criados
        adicionando os sufixos \texttt{\_ta} e \texttt{\_ti} e \texttt{\_tv}
        ao nome original do parâmetro, respectivamente.
\end{smallitem}

\subsection{Exemplo de código}

\begin{verbatim}
int t[10], output[10]
output = map(lambda int x -> x + 2, t)
\end{verbatim}

\subsection{Saída esperada}

\begin{verbatim}
int array: t (size: 10), output (size: 10)
int array fun: t_map (params: int array t)
  int fun: lambda (params: int x)
    ret + x 2
  int var: t_ti
  int array: t_ta (size: 10)
  for: = t_ti 0, < t_ti [len] t, = t_ti + t_ti 1
  do:
    = [index] t_ta t_ti lambda[1 params] [index] t t_ti
  ret t_ta
= output t_map[1 params] t

\end{verbatim}

\section{Suporte a caracteres e cadeias destes}

\subsection{Definições}

Os seguintes recursos são adicionados à linguagem:

\begin{smallitem}
    \item suporte a algumas operações com caracteres ASCII entre $32_{10}$
        (espaço) e $126_{10}$ (\texttt{\~}), inclusive;
    \item variáveis com apenas um caractere, de tipo \texttt{char}, podem ser
        declaradas com aspas simples;
    \item variáveis com múltiplos caracteres, de tipo \texttt{str} podem ser
        declaradas com aspas duplas. Considera-se que um \texttt{str} é
        similar a um \texttt{char array};
    \item o operador de adição causa coerção se utilizado entre \texttt{char} e
        \texttt{char}, ou \texttt{char} e \texttt{str} (independente da ordem);
    \item os operadores de comparação (maior, menor...) podem ser utilizados
        entre estes dois tipos;
    \item o operador de índice pode ser utilizado em um \texttt{str}; esta
        variável é tratada como um \texttt{char};
    \item comentários podem ser feitos em linhas individuais começando com
        \texttt{\#}. Os caracteres válidos são os mesmos citados acima.
        Serão ignorados pelo \emph{parser} e não serão mostrados na
        representação intermediária de código.
\end{smallitem}

\subsection{Exemplo de código}

\begin{verbatim}
    char let = 'a'
    str wd = "luka", result
    # i can add letters together
    result = wd + let
    let = result[1]
\end{verbatim}

\subsection{Saída esperada}

\begin{verbatim}
    char var: let = 'a'
    str var: wd = "luka", result
    = result + wd [str] let
    = let [index] result 1
\end{verbatim}

\section{Geração de código intermediário com LLVM}

\subsection{Roteiro}
Propõe-se compilar código-fonte de Łukasiewicz para representação intermediária
de LLVM (LLVM IR). A fragmentação desta tarefa será definida de forma similar
ao Projeto 1:

\begin{smallenum}
    \item linguagem introdutória
    \item ponto flutuante e valores booleanos
    \item conversão de tipos
    \item caracteres e cadeias
    \item expressões condicionais
    \item laços
    \item múltiplos escopos
    \item funções (usuais, anônimas e de alta ordem)
    \item arranjos
    \item ponteiros
\end{smallenum}

\section{\emph{Transpiler} de Łukasiewicz para Python}

\subsection{Roteiro}

Adicionalmente, propõe-se compilar código-fonte de Łukasiewicz para código
Python, de modo a testar seu funcionamento e ver resultados reais sem a
implementação de um interpretador completo. O roteiro é similar ao anterior. As
características diferentes das linguagens fonte e destino devem ser levadas em
conta, e assim, alguns comportamentos poderão ser simulados de modo diferente:

\begin{smallitem}
    \item tipagem dinâmica em Python faz com que as variáveis não precisem ser
        declaradas junto a seus tipos;
    \item as palavras reservadas \texttt{True} e \texttt{False}, em Python,
        podem ser operadas como inteiros de valor \texttt{0} e \texttt{1};
    \item funcionamento de escopo léxico deve ser retrabalhado para que
        variáveis sejam terminadas ao fim de um bloco lógico (e.g. \texttt{if});
    \item outros comportamentos serão documentados se descobertos.
\end{smallitem}

\end{document}
