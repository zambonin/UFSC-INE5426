\documentclass{beamer}

\usepackage[brazil]{babel}
\usepackage[T1]{fontenc}
\usepackage[utf8]{inputenc}

\usetheme{CambridgeUS}
\usecolortheme{seagull}
\setbeamertemplate{headline}{}
\setbeamertemplate{itemize items}[triangle]
\setbeamertemplate{footline}[frame number]{}
\setbeamertemplate{navigation symbols}{}

\title{
    Novas funcionalidades para Łukasiewicz
}
\author{
    Douglas Martins\thanks{arquiteto de sistema}
    \and Gustavo Zambonin\thanks{gerente de projeto, projetista de linguagem}
    \and Marcello Klingelfus\thanks{testador}
}
\institute{
    Universidade Federal de Santa Catarina \\
    Departamento de Informática e Estatística \\
    INE5426 - Construção de Compiladores
}
\date{}

\begin{document}

\begin{frame}
    \titlepage
\end{frame}

\begin{frame}
    \frametitle{Primeira extensão: o paradigma funcional}

    \begin{itemize}
        \item Função anônima simples --
            {\footnotesize \texttt{lambda type args -> expr}}
        \item Função de alta ordem de mapeamento --
            {\footnotesize \texttt{map(function, list)}}
        \begin{itemize}
            \item aplica função em todos os elementos de um iterável
        \end{itemize}
        \item Função de alta ordem de redução --
            {\footnotesize \texttt{fold(function, list)}}
        \begin{itemize}
            \item reduz elementos de um iterável para um
                tipo primitivo de acordo com uma função
        \end{itemize}
        \item Função de alta ordem de seleção --
            {\footnotesize \texttt{filter(function, list)}}
        \begin{itemize}
            \item remove elementos de um iterável que não respeitam uma
                função booleana
        \end{itemize}
        \item Operadores de tamanho e inserção em arranjos
        \item Extensão da cobertura de erros semânticos
        \item Exemplos em \texttt{test/valid/functional/*.in}
    \end{itemize}

\end{frame}

\begin{frame}[fragile]
    \frametitle{Utilização de funções de alta ordem: \texttt{fold}}

\begin{verbatim}
int ref t[10]
int ref output
output = fold(lambda int ref x, y -> x + y, t)

...
int ref fun: t_fold (params: int ref array t)
  int ref fun: lambda (params: int ref x, int ref y)
    ret + x y
  int ref var: t_tv
  = t_tv [index] t 0
  int var: t_ti
  for: = t_ti 1, < t_ti [len] t, = t_ti + t_ti 1
  do:
    = t_tv + t_tv lambda[2 params] t_tv [index] t t_ti
  ret t_tv
= output t_fold[1 params] t
\end{verbatim}

\end{frame}

\begin{frame}[fragile]
    \frametitle{Utilização de funções de alta ordem: \texttt{filter}}

\begin{verbatim}
int t[10], output[10]
output = filter(lambda int x -> x > 10, t)

...
int array fun: t_filter (params: int array t)
  bool fun: lambda (params: int x)
    ret > x 10
  int var: t_ti
  int array: t_ta (size: 0)
  for: = t_ti 0, < t_ti [len] t, = t_ti + t_ti 1
  do:
    if: lambda[1 params] [index] t t_ti
    then:
      [append] t_ta [index] t t_ti
  ret t_ta
= output t_filter[1 params] t
\end{verbatim}

\end{frame}

\begin{frame}[fragile]
    \frametitle{Segunda extensão: os tipos \texttt{char} e \texttt{char[]}}

    \begin{itemize}
        \item \texttt{char} tem aspas simples, \texttt{char[]} tem aspas duplas
        \item \texttt{char + char}, \texttt{char + char[]} produz coerção
        \item Truncamento de palavras maiores que o tamanho do arranjo declarado
        \item Adiciona também fácil suporte a comentários de uma linha com \#
        \item Exemplos em \texttt{test/valid/strings/*.in}
    \end{itemize}

\begin{verbatim}
char let = 'a'          | char var: let = 'a'
char wd[4], result[10]  | char array: wd (size: 4), ...
wd = "luka"             | = wd "luka"
result = wd + let       | = result + wd [word] let
let = result[1]         | = let [index] result 1
# end of input          |
\end{verbatim}

\end{frame}

\begin{frame}[fragile]
    \frametitle{\emph{Transpiling} para Python}

    \begin{itemize}
        \item Exemplo de tradução entre linguagens:

\begin{verbatim}
int fun math() {        |   def math():
  int z = 0, y = 2, x   |     z = 0
  x = y * 10 + x - 15   |     y = 2
  z = x                 |     x = (((y * 10) + x) - 15)
  ret x                 |     z = x
}                       |     return x
                        |
\end{verbatim}

        \item \texttt{src/scope\_manager.py} é importado nos arquivos
            de saída para simular corretamente os escopos

        \item Outros exemplos em \texttt{test/valid/**/*.py}
    \end{itemize}

\end{frame}

\end{document}
