\documentclass{article}

\usepackage[brazil]{babel}
\usepackage[T1]{fontenc}
\usepackage[a4paper, margin=1.5cm]{geometry}
\usepackage[colorlinks, urlcolor=blue, citecolor=red, linkcolor=black]{hyperref}
\usepackage[utf8]{inputenc}
\usepackage[normalem]{ulem}
\usepackage{enumitem}

\newenvironment{smallenum}{
    \vspace{-1mm}
    \begin{enumerate}[label=\roman*.]
    \setlength{\parskip}{0pt}
    \setlength{\itemsep}{2pt}
}{
    \vspace{-2mm}
    \end{enumerate}
}

\title{\textbf{Especificações da linguagem Łukasiewicz estendida}}
\author{
    Laércio Lima Pilla \\
    {\small criador da linguagem} \vspace{3mm} \\
    Douglas Martins \\
    {\small arquiteto de sistema} \\
    \and Gustavo Zambonin \\
    {\small gerente de projeto, projetista de linguagem} \vspace{3mm} \\
    Marcello Klingelfus \\
    {\small testador} \\
    \and {\small Construção de Compiladores (UFSC -- INE5426)}\footnote{
        \texttt{laercio.pilla@ufsc.br, 
        \{marcelino.douglas,gustavo.zambonin,marcello.klingelfus\}@grad.ufsc.br
        \hfill \href{https://github.com/zambonin/ufsc-ine5426}{src/}
    }}
}
\date{}

\begin{document}

\maketitle

\tableofcontents

\newpage

\section{Linguagem introdutória}

A versão introdutória da linguagem tem as seguintes características:

\begin{smallenum}

\item Um tipo inteiro \texttt{int} com valores representados por números
    naturais compostos por dígitos decimais.

\item Quatro operadores binários: soma \texttt{+}, subtração \texttt{-},
    multiplicação \texttt{*} e divisão \texttt{/}.

\item Um operador menos unário \texttt{-}.

\item Parênteses \texttt{()}.

\item Variáveis com nomes começados por uma letra maiúscula ou minúscula,
    podendo ser seguidos por outros caracteres, dígitos decimais e ‘\_’.

\item Atribuições \texttt{=}.

\end{smallenum}

As diferentes instruções da linguagem são separadas apenas pela marcação de
fim de linha. Múltiplas variáveis de mesmo tipo podem ser declaradas em uma
mesma linha, necessitando-se apenas da separação entre seus nomes com uma
vírgula \texttt{,}. Cada variável pode receber uma atribuição de valor inicial
de valores constantes. A precedência entre os operadores é a mesma da
matemática padrão. Não é possível fazer múltiplas atribuições em uma mesma
linha.

\subsection{Exemplo de código}

\begin{verbatim}
int a_
int    BB, c
int d=0,        e1=1
a_ = d+2*3
BB = (-a_)/  12-1
c = e1*e1/a_
\end{verbatim} 

\subsection{Saída do compilador}

\begin{verbatim}
int var: a_
int var: BB, c
int var: d = 0, e1 = 1
= a_ + d * 2 3
= BB - / -u a_ 12 1
= c * e1 / e1 a_
\end{verbatim}

\subsection{Saída do \emph{transpiler}}

\begin{verbatim}
d = 0
e1 = 1
a_ = (d + (2 * 3))
BB = (-a_ / 12) - 1)
c = (e1 * e1) / a_)
\end{verbatim}

\subsection{Tratamento de erros}

\begin{smallenum}

\item No caso de erros léxicos (símbolos e palavras desconhecidas), o
    compilador deve apenas emitir uma mensagem de erro informando sobre o
    caractere ou palavra desconhecida, não passando nenhum token para a
    análise sintática.

\begin{verbatim}
int a&
[Line 1] lexical error: unknown symbol &
int #####a
[Line 2] lexical error: unknown symbol #####
\end{verbatim}

\item No caso de um erro sintático, o compilador deve ignorar todo o bloco
    de código comprometido (declaração de variáveis, atribuições etc.). A
    mensagem deve seguir similar a original. \uline{O compilador
    poderá emitir erros sintáticos verbosos, apresentando maneiras para
    programar o bloco de código corretamente.}
    
\begin{verbatim}
int 10b
[Line 1] syntax error, unexpected INT, expected FUN
\end{verbatim}

\item No caso de uso de uma variável ainda não declarada, uma mensagem de
    erro semântico deve ser impressa.

\begin{verbatim}
A = 2
[Line 1] semantic error: undeclared variable A
\end{verbatim}

\item No caso de uma variável ser declarada novamente, a declaração
    original deve ser mantida e a nova declaração deve ser ignorada.
    Uma mensagem de erro deve ser emitida.

\begin{verbatim}
int a, a
[Line 1] semantic error: re-declaration of variable a
\end{verbatim}

\end{smallenum}

\subsection{Saída padrão}

\begin{verbatim}
nameT: [a-zA-Z][a-zA-Z0-9_]*
intT: [0-9]+
typeT: int
typeT var: nameT {= intT}?{, nameT {= intT}?}*
\end{verbatim}

\subsection{Saída de erros padrão}

Note que \texttt{<>} pode representar qualquer \emph{token} interno da
ferramenta \texttt{bison}.

\begin{verbatim}
[Line N] lexical error: unknown symbol X
[Line N] syntax error, unexpected <>, expecting <>{ or <>}*
[Line N] semantic error: undeclared variable X
[Line N] semantic error: re-declaration of variable X
\end{verbatim}

\section{Ponto flutuante e valores booleanos}

Esta versão adiciona as seguintes características à linguagem:

\begin{smallenum}

\item Tipo ponto flutuante \texttt{float} com valores representados por
    números reais compostos por dígitos decimais e uma vírgula.

\item Tipo booleano \texttt{bool} com valores \texttt{true} e
    \texttt{false}.

\item Operações unárias e binárias sobre valores de ponto flutuante.

\item Operadores relacionais para comparação entre inteiros ou valores de
    ponto flutuante igual \texttt{==}, diferente \texttt{!=},
    maior \texttt{>}, menor \texttt{<}, maior ou igual \texttt{>=} e
    menor ou igual \texttt{<=}.

\item Operadores binários booleanos e \texttt{\&} e ou \texttt{|}.

\item Operador unário booleano de negação \texttt{!}.

\end{smallenum}

Esta versão não possibilita a conversão entre tipos. Operações relacionais
geram valores booleanos. A precedência dos operadores para valores inteiros e
de ponto flutuante é a padrão. A precedência para expressões booleanas é de
operador unário > operadores relacionais > operadores binários.

\subsection{Exemplo de código}

\begin{verbatim}
float f=1.0, g=0., h= .10, i
bool b = true
i = -f*g-h/2.1
b = ! (i > 0.0) | (i < -2.3)
\end{verbatim}

\subsection{Saída do compilador}

\begin{verbatim}
float var: f = 1.0, g = 0., h = .10, i
bool var: b = true
= i - * -u f g / h 2.1
= b | ! > i 0.0 < i –u 2.3
\end{verbatim}

\subsection{Saída do \emph{transpiler}}

\begin{verbatim}
f = 1.0
g = 0.
h = .10
b = True
i = (-f * g) - (h / 2.1))
b = (not (i > 0.0)) | (i < -2.3))
\end{verbatim}

\subsection{Tratamento de erros}

\begin{smallenum}

\item No caso de tipos incompatíveis usados para uma operação, uma
    mensagem de erro semântico deve ser emitida. No caso de operações
    aritméticas ou atribuições, o tipo esperado é o tipo do operando à
    esquerda. Para reduzir o encadeamento de erros, o tipo retornado por
    essas operações deve ser o tipo do operando à esquerda. Operações
    relacionais e booleanas retornam sempre booleanos. Em caso de erro, o
    menos unário retorna o tipo inteiro.

\begin{verbatim}
int a = 1.0
[Line 1] semantic error: attribution operation expected integer but received float
a = a + true
[Line 2] semantic error: addition operation expected integer but received boolean
\end{verbatim}

\end{smallenum}

\subsection{Saída padrão}

\begin{verbatim}
typeT: int | float | bool
boolT: true | false
\end{verbatim}

\subsection{Saída de erros padrão}

\begin{verbatim}
op: addition | subtraction | multiplication | division | attribution |
    unary minus | equal | different | greater than | less then |
    greater or equal than | less or equal than | and | or | negation
type: integer | float | boolean
[Line N] semantic error: op operation expected type but received type
\end{verbatim}

\section{Conversão de tipos}

Esta versão adiciona as seguintes características à linguagem:

\begin{smallenum}

\item Coerção de valores \texttt{int} para valores \texttt{float} em
    operações binárias com outros valores \texttt{float}.

\item \emph{Casting} entre quaisquer tipos básicos através das operações
    unárias \texttt{[int]}, \texttt{[float]} e \texttt{[bool]}.

\end{smallenum}

As operações de \emph{casting} têm a menor precedência possível.

\subsection{Exemplo de código}

\begin{verbatim}
int i = 0, j
float f = 1.1
bool b = true
j = [int] [int] i + f
i = [int] j
b = b & [bool] f
f = ([float] b) + 0.0
\end{verbatim}

\subsection{Saída do compilador}

\begin{verbatim}
int var: i = 0, j
float var: f = 1.1
bool var: b = true
= j [int] [int] + [float] i f
= i [int] j
= b & b [bool] f
= f + [float] b 0.0
\end{verbatim}

\subsection{Saída do \emph{transpiler}}

\begin{verbatim}
i = 0
f = 1.1
b = True
j = int(int(float(i) + f)))
i = int(j)
b = (b & bool(f))
f = (float(b) + 0.0)
\end{verbatim}

\subsection{Saída padrão}

\begin{verbatim}
[typeT]
\end{verbatim}

\section{Expressões condicionais}

Esta versão inclui estruturas de execução condicional \texttt{if-then} e
\texttt{if-then-else}. O teste da expressão condicional recebe um valor ou
variável booleana. Chaves \texttt{\{\}} são usadas para marcar o início da
região \texttt{then}. No caso da presença da cláusula \texttt{else}, ela deve
vir na mesma linha que o fechamento de chaves do \texttt{then}. Nada mais deve
constar nessas linhas. Na saída padrão, o corpo de cada cláusula altera a
tabulação em dois espaços em branco cumulativos. Os corpos das expressões
condicionais podem ser vazios e podem incluir a declaração de novas variáveis.

\subsection{Exemplo de código}

\begin{verbatim}
int a = 0, b = 1, c, d
  bool teste_falso = false
if a > b
then {
if (a > 0)
then {
c = 10
}

}

if teste_falso
then {
d = 0

} else {
d = 20
}
\end{verbatim}

\subsection{Saída do compilador}

\begin{verbatim}
int var: a = 0, b = 1, c, d
bool var: teste_falso = false
if: > a b
then:
  if: > a 0
  then:
    = c 10
if: teste_falso
then:
  = d 0
else:
  = d 20
\end{verbatim}

\subsection{Saída do \emph{transpiler}}

\begin{verbatim}
a = 0
b = 1
teste_falso = False
if (a > b):
  if (a > 0):
    c = 10
if teste_falso:
  d = 0
else:
  d = 20
\end{verbatim}

\subsection{Tratamento de erros}

\begin{smallenum}

\item Testes devem ser valores booleanos. No caso de um tipo diferente no
    resultado do teste, um erro semântico deve ser gerado.
    
\begin{verbatim}
int a = 0
if a
[Line 2] semantic error: test operation expected boolean but received integer
\end{verbatim}

\end{smallenum}

\subsection{Saída padrão}

\begin{verbatim}
if:
then:
  ...
{else:
  ...
}?
\end{verbatim}

\subsection{Saída de erros padrão}

\begin{verbatim}
op: op | test
\end{verbatim}

\section{Laços}

Esta versão inclui laços for em um estilo similar à linguagem C. Os laços
seguem a estrutura for inicialização, teste, iteração. O trecho de teste é
obrigatório, enquanto os outros trechos são opcionais. A inicialização e a
iteração podem conter apenas uma atribuição. O teste pode conter
\uline{qualquer expressão, desde que booleana}. O corpo do laço é definido por
chaves, pode ser vazio e pode conter a declaração de novas variáveis. 
Na saída padrão, o corpo de cada laço altera a tabulação em dois espaços
em branco cumulativos.

\subsection{Exemplo de código}

\begin{verbatim}
int i, j = 0 
for , j < 10 , j = j + 2 {
}
for i = 0, i < 10   , i = i + 1 {
int temp
temp = j + i
j = temp
}
j = j + 0
\end{verbatim}

\subsection{Saída do compilador}

\begin{verbatim}
int var: i, j = 0
for: , < j 10, = j + j 2
do:
for: = i 0, < i 10, = i + i 1
do:
  int var: temp
  = temp + j i
  = j temp
= j + j 0
\end{verbatim}

\subsection{Saída do \emph{transpiler}}

\begin{verbatim}
j = 0
while (j < 10):
  j = (j + 2)
i = 0
while (i < 10):
  temp = (j + i)
  j = temp
  i = (i + 1)
j = (j + 0)
\end{verbatim}

\subsection{Tratamento de erros}

Os testes dos laços devem respeitar às mesmas regras que os testes de
expressões condicionais e devem gerar os mesmos erros.

\subsection{Saída padrão}

\begin{verbatim}
for: , ,
do:
    ...
\end{verbatim}

\section{Múltiplos escopos}

Esta versão adiciona a diferenciação entre escopos. A linguagem trabalha com
um escopo global e um novo escopo interno para cada corpo de \texttt{then},
\texttt{else} e \texttt{for}. Dentro de novos escopos, novas variáveis com o
mesmo nome de variáveis de escopos externos podem ser declaradas. O tempo de
vida das variáveis fica também atrelado à duração do escopo onde elas são
declaradas.

\subsection{Exemplo de código}

\begin{verbatim}
int i
if true {
  float i = 0.0
}
for i = 0, i < 2, i = i + 2 {
  int a
}
bool a = true
\end{verbatim}

\subsection{Saída do compilador}

\begin{verbatim}
int var: i
if: true
then:
  float var: i = 0.0
for: = i 0, < i 2, = i + i 2
do:
  int var: a
bool var: a = true
\end{verbatim}

\subsection{Saída do \emph{transpiler}}

\begin{verbatim}
exec(open('../../../src/scope_manager.py', 'r').read())
s_context()
if True:
  i = 0.0
r_context()
s_context()
i = 0
while (i < 2):
  i = (i + 2)
r_context()
a = True
\end{verbatim}

\subsection{Tratamento de erros}

\begin{smallenum}

\item Variáveis somente são visíveis dentro de seus escopos e seus escopos
    filhos, levando a erros semânticos como listados no item 1.3.3 no caso
    de tentativas indevidas de uso de variáveis.

\end{smallenum}

\section{Funções}

Esta versão adiciona as seguintes características relacionadas a funções:

\begin{smallenum}

\item Funções podem ser declaradas para uso posterior no código.

\item Funções podem ser definidas, o que envolve a declaração da assinatura da
    função e de seu corpo.

\item Funções podem ser usadas em testes e atribuições.

\item Funções podem ser declaradas e definidas nos mesmos lugares onde
    variáveis podem.

\item Funções têm zero ou mais parâmetros.

\item Ao fim do código, todas as funções declaradas precisam ter sido
    definidas.

\item Todas as funções terminam com uma chamada de retorno com a palavra
    reservada \texttt{ret}.

\item Os nomes de funções seguem as mesmas regras que os nomes de variáveis.

\item Os corpos de funções possuem os próprios escopos.

\item Os parâmetros da função funcionam de forma similar a variáveis
    declaradas no corpo da função.

\item Declarações de funções não geram mensagens de saída.

\item Na saída padrão, o corpo de cada função altera a tabulação em dois
    espaços em branco cumulativos.

\end{smallenum}

\subsection{Exemplo de código}

\begin{verbatim}
bool fun f ()
bool fun f () {
ret false
}
if f() {
  int a = 0
  int fun f2 ( int x ) {
          int a
         a = x + 1
         ret a
  }
  a = f2 ( a)
}
\end{verbatim}

\subsection{Saída do compilador}

\begin{verbatim}
int fun: f (params: )
  ret false
if: f[0 params]
then:
  int var: a = 0
  int fun: f2 (params: int x)
    int var: a
    = a + x 1
    ret a
  = a f2[1 params] a
\end{verbatim}

\subsection{Saída do \emph{transpiler}}

\begin{verbatim}
exec(open('../../../src/scope_manager.py', 'r').read())
def f():
  return False

s_context()
if f():
  a = 0
  def f2(x):
    a = (x + 1)
    return a

  a = f2(a)
r_context()
\end{verbatim}

\subsection{Tratamento de erros}

\begin{smallenum}
    
\item Uma função deve ter apenas um retorno, o qual deve ser feito na
    última linha de seu corpo. A ausência de um retorno ou a presença de
    outras linhas de código deve gerar um erro de sintaxe.

\item \uline{Uma função com tipo de retorno diferente do tipo de sua
    declaração deve gerar um erro semântico.}

\begin{verbatim}
int fun f() {
  ret false 
}
[Line 3] semantic error: function f has incoherent return type
\end{verbatim}

\item Uma função declarada mas nunca definida até o fim de seu escopo deve
    gerar uma mensagem de erro semântico.
    
\begin{verbatim}
fool fun myfun()
^D
[Line 2] semantic error: function myfun is declared but never defined
\end{verbatim}

\item No caso de redeclaração ou redefinição de uma função em seu mesmo
    escopo original, um erro semântico deve ser gerado no mesmo estilo do
    item 1.3.4.

\begin{verbatim}
int fun f()
int fun f()
[Line 3] semantic error: re-definition of function f
\end{verbatim}

\item No caso de um parâmetro de tipo incompatível, um erro semântico deve
    ser emitido.

\begin{verbatim}
int a
int fun f (int x, int y)
a = f (0.0, 0)
[Line 3] semantic error: parameter x expected integer but received float
\end{verbatim}

\item No caso de parâmetros faltantes ou excesso de parâmetros em uma
    chamada de função, erros semânticos devem ser gerados.

\begin{verbatim}
bool b
bool fun x (int a__)
b = x()
[Line 3] semantic error: function x expects 1 parameters but received 0
b = x(1, 2, 3)
[Line 3] semantic error: function x expects 1 parameters but received 3
\end{verbatim}

\item No caso de discordância sobre o número de parâmetros entre a declaração
    e a definição da função, um erro de redefinição deve ser gerado como no
    item 7.3.3. O mesmo deve ser feito para o caso de discordância entre os
    nomes do parâmetros.

\end{smallenum}

\subsection{Saída padrão}

\begin{verbatim}
typeT fun: nameT (params: {typeT nameT{, typeT nameT}*}?)
ret
nameT[N params]
\end{verbatim}

\subsection{Saída de erros padrão}

\begin{verbatim}
[Line N] semantic error: function nameT has incoherent return type
[Line N] semantic error: function nameT is declared but never defined
[Line N] semantic error: parameter nameT expected typeT but received typeT
[Line N] semantic error: function nameT expects N parameters but received N
\end{verbatim}

\section{Arranjos}

Esta versão adiciona as seguintes características relacionadas a arranjos:

\begin{smallenum}

\item Arranjos de tipos básicos podem ser declarados \uline{com qualquer
    tamanho não-negativo.}

\item Índices de arranjos são sempre valores inteiros.

\item \uline{Declarações de arranjos e acesso a seus elementos são feitos
    utilizando colchetes.}

\item \uline{O operador unário de tamanho \texttt{[len]} é suportado, com
    precedência igual aos operadores de coerção.}

\item \uline{O operador de inserção \texttt{<-} é suportado, aumentando
    o tamanho do arranjo em $1$ na sua utilização.}

\end{smallenum}

\subsection{Exemplo de código}

\begin{verbatim}
int a[10]
bool parity[10]
int i
for i = 0, i < 10, i = i + 1 {
a[i] = i
if a[i] / 2  == 0 {
    parity[i] = true
} else {
    parity[i] = false
}
}
\end{verbatim}

\subsection{Saída do compilador}

\begin{verbatim}
int array: a (size: 10)
bool array: parity (size: 10)
for: = i 0, < i 10, = i + i 1
do:
  = [index] a i i
 if: == / [index] a i 2 0
  then:
    = [index] parity i true
  else:
    = [index] parity i false
\end{verbatim}

\subsection{Saída do \emph{transpiler}}

\begin{verbatim}
exec(open('../../../src/scope_manager.py', 'r').read())
a = [0] * 10
parity = [0] * 10
s_context()
i = 0
while (i < 10):
  a[i] = i
  s_context()
  if ((a[i] / 2) == 0):
    parity[i] = True
  else:
    parity[i] = False
  r_context()
  i = (i + 1)
r_context()
\end{verbatim}

\subsection{Tratamento de erros}

\begin{smallenum}

\item No caso de um índice de um tipo outro que inteiro, um erro semântico
    deve ser gerado.

\begin{verbatim}
int a[10]
a[false] = 1
[Line 2] semantic error: index operation expects integer but received boolean
\end{verbatim}

\item \uline{O operador de inserção espera um tipo compatível com o tipo do
    arranjo. Caso contrário, um erro semântico deve ser gerado.}

\begin{verbatim}
bool b[10]
b <- 1.2
[Line 3] semantic error: append operation expected boolean but received float
\end{verbatim}

\item No caso da declaração de um arranjo com um tamanho que não seja um
    valor inteiro, um erro de sintaxe deve ser gerado.

\item \uline{Operações onde o tamanho do arranjo à direita é maior do que
    o tamanho do arranjo à esquerda devem gerar um erro semântico.}

\begin{verbatim}
int a[9], b[10]
bool c
c = a != b
[Line 4] semantic error: operation between mismatched array sizes
\end{verbatim}

\item \uline{Os operadores de índice e inserção esperam um arranjo no lado
    esquerdo da operação.}

\begin{verbatim}
int a, b
b = a[1]
[Line 2] semantic error: left hand side of index operation is not an array
\end{verbatim}

\item \uline{O operador unário de tamanho do arranjo espera um arranjo.}

\begin{verbatim}
int a, b
b = [len] a
[Line 4] semantic error: length operation expects an array
\end{verbatim}

\end{smallenum}

\subsection{Saída padrão}

\begin{verbatim}
typeT array: nameT (size: N)
[index]
[len]
[append]
\end{verbatim}

\subsection{Saída de erros padrão}

\begin{verbatim}
op: op | index | length | append
[Line N] semantic error: index operation expects integer but received { boolean | float }
[Line N] semantic error: operation between mismatched array sizes
[Line N] semantic error: left hand size of { index | append } operation is not an array
[Line N] semantic error: length operation expects an array
\end{verbatim}

\section{Ponteiros}

Esta versão inclui a declaração e uso de ponteiros sobre os tipos básicos,
arranjos e ponteiros. Ponteiros são identificados pela palavra reservada
\texttt{ref}. O \texttt{lvalue} de qualquer variável pode ser acessado pelo
operador \texttt{addr}. Referências não são feitas para arranjos nem funções,
sendo usadas apenas para variáveis e itens de arranjos. Ponteiros podem ser
usados como parâmetros de funções e retorno de funções.

\subsection{Exemplo de código}

\begin{verbatim}
int a[10]
int i = 0
int ref mypointer
int ref pointers[2]
mypointer = addr i
i = ref mypointer + 1
pointers[0] = mypointer
pointers[1] = addr a[3]
int ref ref doublepointer
doublepointer = addr pointers[0]
\end{verbatim}

\subsection{Saída do compilador}

\begin{verbatim}
int array: a (size: 10)
int var: i = 0
int ref var: mypointer
int ref array: pointers (size: 2)
= mypointer [addr] i
= i + [ref] mypointer 1
= [index] pointers 0 mypointer
= [index] pointers 1 [addr] [index] a 3
int ref ref var: doublepointer
= doublepointer [addr] [index] pointers 0
\end{verbatim} 

\subsection{Saída do \emph{transpiler}}

\begin{verbatim}
exec(open('../../../src/scope_manager.py', 'r').read())
a = [0] * 10
i = 0
pointers = [0] * 2
mypointer = i
i = (mypointer + 1)
pointers[0] = mypointer
pointers[1] = a[3]
doublepointer = pointers[0]
\end{verbatim}

\subsection{Tratamento de erros}

\begin{smallenum}

\item No caso de operações binárias com tipos diferentes, erros semânticos
    devem ser emitidos.

\begin{verbatim}
int i = 0
int ref point
point = i
[Line 3] semantic error: attribution operation expects integer pointer but received integer
\end{verbatim}

\item O operador unário \texttt{ref} espera um valor que seja ponteiro de
    algo. Caso o valor não seja um ponteiro, um erro semântico deve ser
    disparado.

\begin{verbatim}
int i
i = ref 0
[Line 2] semantic error: reference operation expects a pointer
i = ref i
[Line 3] semantic error: reference operation expects a pointer
\end{verbatim}
    
\item O operador unário \texttt{addr} espera uma variável ou item de
    arranjo. Caso contrário, um erro semântico deve ser gerado.

\begin{verbatim}
bool ref b
b = addr true
[Line 0] semantic error: address operation expects a variable or array item
\end{verbatim}

\end{smallenum}

\subsection{Saída padrão}

\begin{verbatim}
typeT {ref}* var: nameT
[addr]
[ref]
\end{verbatim}

\subsection{Saída de erros padrão}

\begin{verbatim}
[Line N] semantic error: op operation expects typeT {pointer}* but received typeT {pointer}*
[Line N] semantic error: reference operation expects a pointer
[Line N] semantic error: address operation expects a variable or array item
\end{verbatim}

\section{Caracteres e cadeias destes}

Esta versão adiciona as seguintes características à linguagem:

\begin{smallenum}

\item Tipo de caractere \texttt{char} com possíveis valores ASCII no intervalo
    $32_{10}$ (espaço) e $126_{10}$ (\texttt{\~}), inclusive, exceto aspas
    simples (\texttt{'}) e duplas (\texttt{"}).

\item Um \texttt{char} pode ser declarado utilizando aspas simples, e não
    pode ser vazio.

\item Um arranjo de \texttt{char} pode ser declarado utilizando aspas duplas,
    e pode ser a palavra vazia.

\item Acontecerá coerção para arranjos de \texttt{char} com o operador
    \texttt{[word]} na operação entre dois caracteres, ou um caractere e um
    arranjo de caracteres.

\item Palavras com valores além da capacidade de seus tipos (1 para
    \texttt{char}, $n$ para \texttt{char array}) serão truncadas.

\item Suporte a comentários de linha única, que iniciam com o símbolo
    \texttt{\#}, em qualquer parte da linha.

\item Não existem restrições quanto às operações entre caracteres e cadeias.

\end{smallenum}

\subsection{Exemplo de código}

\begin{verbatim}
char let = 'a'
char wd[4], result[10]
wd = "luka"
# i can add letters together
result = wd + let
let = result[1]
\end{verbatim}

\subsection{Saída do compilador}

\begin{verbatim}
char var: let = 'a'
char array: wd (size: 4), result (size: 10)
= wd "luka"
= result + wd [word] let
= let [index] result 1
\end{verbatim}

\subsection{Saída do \emph{transpiler}}

\begin{verbatim}
exec(open('../../../src/scope_manager.py', 'r').read())
let = 'a'
wd = [0] * 4
result = [0] * 10
wd = "luka"
result = (wd + str(let))
let = result[1]
\end{verbatim}

\subsection{Tratamento de erros}

\begin{smallenum}

\item Caso um caractere seja declarado com mais de uma letra, seu valor deve
    ser truncado para a primeira letra, e uma advertência deve ser gerada.

\begin{verbatim}
char a = 'as'
[Line 2] warning: value truncated to 'a'
\end{verbatim}

\item Uma cadeia de caracteres inicializada com mais letras do que sua
    declaração original deve ser truncada, e uma advertência deve ser gerada.

\begin{verbatim}
char a[5]
a = "too big of a word"
[Line 3] warning: value truncated to "too b"
\end{verbatim}

\end{smallenum}

\subsection{Saída padrão}

Note que todos os comentários serão ignorados pela ferramenta \texttt{flex}.

\begin{verbatim}
charT: [\'][ !#-&(-~]+[\']
wordT: [\"][ !#-&(-~]*[\"]
typeT: typeT | char
type: type | character
[word]
\end{verbatim}

\subsection{Saída de erros padrão}

\begin{verbatim}
[Line N] warning: value truncated to X
\end{verbatim}

\section{Funções anônimas e de alta ordem}

Esta versão adiciona as seguintes características à linguagem:

\begin{smallenum}

\item \texttt{lambda} e a respectiva letra minúscula em grego, $\lambda$
    (\texttt{U+03BB}), tornam-se palavras reservadas da linguagem, bem como
    o símbolo \texttt{->}.

\item Funções anônimas devem conter no mínimo um (1) argumento.

\item Uma função anônima pode ser descartada com a diretiva
    \texttt{$\lambda$()}, para que outra possa ser declarada.

\item \texttt{map}, \texttt{fold} e \texttt{filter} tornam-se palavras
    reservadas da linguagem.

\item Funções de alta ordem devem ter obrigatoriamente dois argumentos,
    nesta ordem: uma função anônima e um arranjo, chamados respectivamente
    de \texttt{$\lambda$} e \texttt{it} neste documento para facilitar a
    visualização das regras.

\item Os tipos dos parâmetros devem ser coerentes com o tipo da função de alta
    ordem.

\item \texttt{map} sempre retornará um arranjo de tipo e tamanho 
    igual a \texttt{it}. Sua função anônima deve ter um (1) parâmetro.

\item \texttt{fold} sempre retornará um elemento de tipo igual ao tipo
    primitivo de \texttt{it}. Sua função anônima deve ter dois (2) parâmetros.

\item \texttt{filter} sempre retornará um arranjo de tipo igual a
    \texttt{it}, mas não necessariamente com o mesmo tamanho. Ademais,
    seu primeiro argumento deve ser sempre uma função anônima booleana
    com um (1) parâmetro.

\item Para evitar erros de redefinição, as funções de alta ordem levam como
    prefixo o nome de seu segundo parâmetro seguido de um traço inferior.

\item Podem ser utilizadas apenas em atribuições para novas variáveis.

\item Como o código que simula o funcionamento das funções de alta ordem
    é gerado automaticamente, erros em cascata são comuns caso as restrições
    não sejam obedecidas.

\end{smallenum}

\subsection{Exemplo de código}
\begin{verbatim}
int ref t[10], output[10]
output = map(lambda int ref x -> x, t)
\end{verbatim}

\subsection{Saída do compilador}

\begin{verbatim}
int ref array: t (size: 10), output (size: 10)
int ref array fun: t_map (params: int ref array t)
  int ref fun: lambda (params: int ref x)
    ret x
  int var: t_ti
  int ref array: t_ta (size: 10)
  for: = t_ti 0, < t_ti [len] t, = t_ti + t_ti 1
  do:
    = [index] t_ta t_ti lambda[1 params] [index] t t_ti
  ret t_ta
= output t_map[1 params] t
\end{verbatim}

\subsection{Saída do \emph{transpiler}}

\begin{verbatim}
exec(open('../../../src/scope_manager.py', 'r').read())
t = [0] * 10
output = [0] * 10
def t_map(t):
  def U+03BB(x):
    return x

  t_ta = [0] * 10
  s_context()
  t_ti = 0
  while (t_ti < len(t)):
    t_ta[t_ti] = U+03BB(t[t_ti])
    t_ti = (t_ti + 1)
  r_context()
  return t_ta

output = t_map(t)
\end{verbatim}

\subsection{Tratamento de erros}

\begin{smallenum}

\item Funções anônimas e de alta ordem com número de parâmetros ilegal
    geram erros sintáticos.

\item Funções de alta ordem cujos \texttt{$\lambda$} têm número de
    parâmetros ilegal geram erros semânticos.

\begin{verbatim}
int t[10], output[10]
output = map(lambda int x, y -> x + 2, t)
[Line 6] semantic error: function lambda expects 2 parameters but received 1
[Line 8] semantic error: map's lambda expects 1 parameters but received 2
...
\end{verbatim}

\item Caso \texttt{it} não seja um arranjo, um erro semântico deve ser gerado.

\begin{verbatim}
int t
int output[10]
output = map(lambda int x -> x + 2, t)
[Line 4] semantic error: high order function's second parameter must be of array type
...
\end{verbatim}

\end{smallenum}

\subsection{Saída padrão}

\begin{verbatim}
funcH: map | fold | filter
func: funcH | lambda
U+03BB
\end{verbatim}

\subsection{Saída de erros padrão}

\begin{verbatim}
[Line N] semantic error: high order function's second parameter must be of array type
[Line N] semantic error: funcH's lambda expects N parameters but received N
\end{verbatim}

\section{\emph{Transpiler} para Python}

Este recurso é adicionado à linguagem, com as seguintes considerações:

\begin{smallenum}

\item Python não necessita de tipos junto à declaração da linguagem, então
    declarações sem inicializações não fazem sentido e podem ser descartadas.

\item Todas as operações são cercadas de parênteses para que seus significados
    sejam o mais próximo possível da linguagem aqui descrita.

\item Todos os códigos gerados iniciam com um cabeçalho que importa um par
    de funções utilizadas automaticamente para que os escopos sejam
    corretamente simulados.

\item Funções declaradas mas não definidas não existem em Python; portanto,
    a palavra chave \texttt{pass} é utilizada para inicializar uma função
    sem qualquer utilidade.

\item Como a ideia de ponteiros não existe em Python, não existe uma possível
	conversão válida. Portanto, todos os operadores \texttt{addr} e
	\texttt{ref} são retirados.

\item Python não contém qualquer estrutura de iteração similar a um
    \texttt{for} estilo C, então todos os \texttt{for} são traduzidos para
    estruturas \texttt{while}.

\end{smallenum}

\end{document}
